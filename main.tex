\documentclass{article}

\usepackage{color}
\usepackage{xcolor}

\newcommand{\setbackgroundcolour}{\pagecolor[rgb]{0,0.168,0.211}}
\newcommand{\settextcolour}{\color[rgb]{0.512,0.578,0.586}}
\newcommand{\darkmode}{\setbackgroundcolour\settextcolour}
\darkmode

\usepackage{hyperref}

\usepackage{tikz}
\usetikzlibrary{calc}
\usetikzlibrary{math}

\newlength{\barwidth}
\setlength{\barwidth}{0.5cm}

\newcommand{\drawaxes}[1]{
  \draw [->] (-.2, 0) -> (4 * \barwidth + .2cm, 0) node[anchor=west] {$x$};
  \draw [->] (-.2, 0) -> (-.2, 5.2) node[anchor=south] {$#1$};
  \foreach \x in {0, ..., 4}
    \draw (\x * \barwidth, -1pt) -- (\x * \barwidth, 1pt);
  \foreach \x/\char in {0.5/$a$, 1.5/$b$, 2.5/$c$, 3.5/$d$}
    \draw (\x * \barwidth, 0) node[anchor=north, text height=8] {\char};
}

\newcommand{\drawbar}[3]{
  \draw[#3] (#1, 0) -- (#1, 12.5cm * #2) -- (\barwidth + #1, 12.5cm * #2) --
  (\barwidth + #1, 0);
}

% \font\nullfont=cmr10
% \tracinglostchars=2
% \tracingcommands=1
% \tracingmacros=1

\newcommand{\drawpmf}{
\begin{tikzpicture}
  \drawaxes{P(x)}
  \foreach \y in {0.125, 0.25, 0.375}{
    \draw (-.2cm - 1pt, 12.5 * \y) node[anchor=east] {$\y$}
      --  (-.2cm + 1pt, 12.5 * \y);
  }
  \drawbar{0\barwidth}{0.125}{}
  \drawbar{1\barwidth}{0.25 }{}
  \drawbar{2\barwidth}{0.375}{}
  \drawbar{3\barwidth}{0.25 }{}
\end{tikzpicture}
}

\newcommand{\drawapprox}[2]{
\begin{tikzpicture}
  \tikzmath{
    \splus = #1 + 1;
    \yunit = 12.5 / \splus;
  }
  \draw[xstep=\barwidth, ystep=\yunit, lightgray, very thin]
    (0, 0) grid (4 * \barwidth, 5.2);
    \drawaxes{s'}
  \foreach \y [parse=true, count=\s] in {\yunit, 2 * \yunit, ..., 5.2}{
    \node (tick) at (-.2cm, \y) {};
    \draw ($(tick) - (1pt, 0)$) -- ($(tick) + (1pt, 0)$);
    \draw ($(tick) - (0, \yunit / 2) - (.05, 0)$) node[anchor=east]
      {\pgfmathparse{int(\s - 1)} #2 $\pgfmathresult$};
  }
  \foreach \x/\freq/\start in {0/1/0, 1/2/1, 2/3/3, 3/2/6}{
    \foreach \y [parse=true, count=\s] in {\yunit, 2 * \yunit, ..., 5.2}{
      \tikzmath{
        \snew = int(8 * int((\s - 1) / \freq) + mod((\s - 1), \freq) + \start);
        if \snew>#1 then {\toprint = \snew;} else {let \toprint = ;};
        }
      \draw ($6.25/\splus*(0, -1) + 0.45*(\barwidth,0)
      + \s*(0, 12.5 / \splus) + \x*(\barwidth, 0)$)
      node [text=lightgray] {#2 $\toprint$};
    }
  }
  \drawbar{0\barwidth}{0.125}{thin, gray}
  \drawbar{1\barwidth}{0.25} {thin, gray}
  \drawbar{2\barwidth}{0.375}{thin, gray}
  \drawbar{3\barwidth}{0.25} {thin, gray}
  \foreach \x/\freq/\start in {0/1/0, 1/2/1, 2/3/3, 3/2/6}{
    \foreach \y [parse=true, count=\s] in {\yunit, 2 * \yunit, ..., 5.2}{
      \tikzmath{
        \snew = int(8 * int((\s - 1) / \freq) + mod((\s - 1), \freq) + \start);
        if \snew<=#1 then {\toprint = \snew;} else {let \toprint = ;};
        }
      \draw ($(0, -\yunit / 2) + 0.47*(\barwidth,0)
      + \s*(0, \yunit) + \x*(\barwidth, 0)$) node {#2 $\mathbf\toprint$};
    }
  }
\end{tikzpicture}
}

\newlength{\intervalwidth}
\setlength{\intervalwidth}{0.7\textwidth}

\newlength{\gap}
\setlength{\gap}{0.15cm}

\newlength{\lineheight}
\setlength{\lineheight}{0.45cm}
\newcommand{\drawinterval}{
\begin{tikzpicture}
  \draw (-30pt, 0) node [anchor=east] {$s$};
  \draw (0, -.5\lineheight) -- (0, .5\lineheight);
  \draw (\intervalwidth, -.5\lineheight) -- (\intervalwidth, .5\lineheight);
  \foreach \s in {1, ..., 7}{
    \draw (\s * 0.125 * \intervalwidth, -.5\lineheight)
      -- (\s * 0.125 * \intervalwidth, .5\lineheight);
  }
  \foreach \y in {0, ..., 7}{
    \tikzmath{
      \x = (\y * 0.125 + 0.0625);
      \r = int(\y + 64);
    }
    \draw (\x\intervalwidth, 0) node {$\r$};
  }

  \draw (-30pt, -\lineheight -\gap) node [anchor=east] {$s \bmod 2^{r_p}$};
  \draw (0, -.5\lineheight - \gap) -- (0, -1.5\lineheight -\gap);
  \draw (\intervalwidth, -.5\lineheight - \gap)
    -- (\intervalwidth, -1.5\lineheight - \gap);
  \foreach \s in {1, ..., 7}{
    \draw (\s * 0.125 * \intervalwidth, -.5\lineheight - \gap)
      -- (\s * 0.125 * \intervalwidth, -1.5\lineheight - \gap);
  }
  \foreach \s in {0, ..., 7}{
    \tikzmath{
      \x = (\s * 0.125 + 0.0625);
    }
    \draw (\x\intervalwidth, -\lineheight - \gap) node  {$\s$};
  }

  \draw (-30pt, -2\gap - 2\lineheight) node [anchor=east] {$x$};
  \draw (0, -2\gap - 1.5\lineheight) -- (0, -2\gap - 2.5\lineheight - 6pt);
  \draw (\intervalwidth, -2\gap - 1.5\lineheight)
    -- (\intervalwidth, -2\gap - 2.5\lineheight - 6pt);
  \draw (0, -2\gap - 2.5\lineheight - 6pt) node [anchor=north] {$0$};
  \draw (\intervalwidth, -2\gap - 2.5\lineheight - 6pt) node [anchor=north] {$1$};
  \draw (0.125  \intervalwidth, -2\gap - 1.5\lineheight)
    -- (0.125 * \intervalwidth, -2\gap - 2.5\lineheight);
  \draw (0.375  \intervalwidth, -2\gap - 1.5\lineheight)
    -- (0.375 * \intervalwidth, -2\gap - 2.5\lineheight);
  \draw (0.75  \intervalwidth, -2\gap - 1.5\lineheight)
    -- (0.75 * \intervalwidth, -2\gap - 2.5\lineheight);
  \draw (0.06125\intervalwidth, -2\gap - 2\lineheight)
    node [text height=1ex] {$a$};
  \draw (0.25   \intervalwidth, -2\gap - 2\lineheight)
    node [text height=1ex] {$b$};
  \draw (0.5625 \intervalwidth, -2\gap - 2\lineheight)
    node [text height=1ex] {$c$};
  \draw (0.875  \intervalwidth, -2\gap - 2\lineheight)
    node [text height=1ex] {$d$};

\end{tikzpicture}
}


\usepackage{booktabs}

\usepackage{amsmath}
\usepackage{amssymb}
\usepackage{mathtools}
\newcommand\given{\,\vert\,}
\DeclarePairedDelimiter\abs{\lvert}{\rvert}

\usepackage[style=authoryear, date=year, natbib=true, maxcitenames=2,
            maxbibnames=10, uniquelist=false]{biblatex}
\setlength\bibitemsep{1.5\itemsep}
\DeclareFieldFormat[article, book, inbook, incollection, inproceedings, misc,
                    thesis, unpublished, techreport, phdthesis]{title}{#1}
\DeclareNameAlias{sortname}{family-given}
\addbibresource{ans-notes.bib}
\renewcommand*{\intitlepunct}{\space}

% TODO: Maybe delete this stuff:
\usepackage{siunitx}
\DeclareSIUnit[parse-numbers=false]\bit{bits}
\newcommand\bits{\,\si\bit}

\usepackage{listings}
\lstset{language=Python, mathescape=true, columns=flexible, xleftmargin=0.9cm,
        basicstyle=\ttfamily, commentstyle=\textrm}

\usepackage{cleveref}
\newtheorem{problem}{Problem}

\newcommand{\push}{\texttt{push}}
\newcommand{\pop}{\texttt{pop}}

\author{
  James Townsend\\
  University College London
}
\title{A tutorial on the range variant of asymmetric numeral systems}

\begin{document}
\maketitle
\section{Introduction}
We are interested in algorithms for lossless compression of sequences of data.
The range variant of asymmetric numeral systems (ANS) is such an algorithm,
and, like arithmetic coding (AC), it is close to optimal in terms of
compression rate \citep{dudaAsymmetricNumeralSystems2009}. The key difference
between ANS and AC is that ANS is last-in-first-out (LIFO), or `stack-like',
while AC is first-in-first-out (FIFO), or `queue-like'.

ANS comprises two functions, which we denote \push\ and \pop, for encoding and
decoding, respectively (the names refer to the analagous stack operations). The
\push\ function accepts some pre-compressed information $m$ (short for
`message'), and a symbol $x$ to be compressed, and returns a piece of
compressed information, $m'$. Thus it has the signiature
\begin{equation}
  \push:(m, x) \mapsto m'.
\end{equation}
The new compressed state, $m'$, contains precisely the same information as the
pair $(m, x)$, and therefore \push\ can be inverted, giving a decoder mapping.
The decoder, \pop, maps from $m'$ back to $m, x$:
\begin{equation}
  \pop:m' \mapsto (m, x).
\end{equation}
The functions \push\ and \pop\ are inverse to one another, so
$\push(\pop(m))=m$ and $\pop(\push(m, x)) = (m, x)$.

Encoding and decoding both require knowledge of some model over symbols. We use
$\mathcal{A} = \{a_1, \ldots, a_I\}$ to denote the alphabet from which symbols
$x$ are drawn. We denote the probability mass function of the model $P$. Later
we will need to assume that all probability masses are quantized to some
precision $r_p$, i.e.\ that there exist integers $p_1, \ldots, p_I$ such that
$P(a_i) = \frac{p_i}{2^{r_p}}$ for each $i = 1,\ldots,I$.

\section{Building an encoder/decoder pair}
The problem which the ANS encoder solves is

\begin{problem}\label{prob:default}
  Given a sequence of random variables $X_1, \ldots, X_N$, find an invertible
  algorithm which will map any sample $x_1, \ldots, x_N$ to a binary message
  $b$, such that the length of $b$ is close to the information content $h(x_1,
  \ldots, x_N)$.
\end{problem}

Given that the algorithm is invertible, we can reformulate \Cref{prob:default}
in terms of the inverse. This leads to a different, but equivalent, problem
statement:

\begin{problem}\label{prob:alt}
  Given a sequence of random variables $X_1, \ldots, X_N$, find an invertible
  algorithm which maps a source of bits to a sequence $x_1, \ldots, x_N$, such
  that the number of bits observed is close to the information content $h(x_1,
  \ldots, x_N)$.
\end{problem}

Our description of the details of ANS focuses on the decoding algorithm,
because this leads to a more straightforward presentation. We describe the
decoder and show that it solves \Cref{prob:alt}, then we show how to invert it
to form an encoder.

The decoding algorithm we describe will be formed from a series of ANS \pop\
operations, its inverse will be formed from ANS \push\ operations.

\subsection*{The structure of the message}
We use a pair $m = (s, t)$ as the data structure for $m$. The element $t$ is a
stack of unsigned integers of some fixed precision $r_t$. This stack has its
own push and pop operations, which we denote $\texttt{stack\_push}$ and
$\texttt{stack\_pop}$ respectively. The element $s$ is an unsigned integer with
precision $r_s$ where $r_s > r_t$. In our implementation we use $r_t = 32$ and
$r_s = 64$. The stack $t$, along with its pop operation, can model the `source
of bits' from our problem statement above.
% TODO: Diagram of $m$

\subsection*{Constructing the pop operation}
Our strategy for performing a decode with \pop\ will be to firstly to extract a
symbol from $s$. We do this using a bijective function $d:\mathbb
N\rightarrow\mathbb N\times\mathcal{A}$, which takes an integer $s$ as input
and returns a pair $(s', x)$, where $s'$ is an integer and $x$ is a symbol.
Thus \pop\ begins
\begin{lstlisting}
def pop($s$, $t$):
    $s'$, $x= d(s)$
\end{lstlisting}
We design the function $d$ so that if $s\geq 2^{r_s - r_t}$, then
\begin{equation}\label{eq:inner-decoder}
  \log s'\geq\log s - h(x) + \log (1 - \epsilon)
\end{equation}
where $\epsilon := 2^{r_p + r_t - r_s}$. We give details of $d$ and prove
\cref{eq:inner-decoder} below. Note that for small $\epsilon$ we have $\log (1
- \epsilon) \approx -\frac{\epsilon}{\ln2}$, and thus this term is small.

After extracting a symbol using $d$, we check whether $s'$ is below $2^{r_s -
r_t}$, and if it is we \texttt{stack\_pop} an integer from $t$ and move its
contents into the lower order bits of $s'$, increasing the size of $s'$. We
refer to this as `renormalization'. Having done this, we return the new message
and the symbol $x$. The full definition of \pop\ is thus
\begin{lstlisting}
def pop($s$, $t$):
    $s'$, $x= d(s)$
    $s$, $t =$ renorm($s'$, $t$)
    return ($s$, $t$), $x$
\end{lstlisting}

Renormalization is necessary to ensure that the $s$ returned is large enough to
ensure that \cref{eq:inner-decoder} holds at the start of any future \pop\
operation. The \texttt{renorm} function has a while loop, which pushes elements
from $t$ into the lower order bits of $s$ until $s$ is full to capacity. To be
precise:

\begin{lstlisting}
def renorm($s$, $t$):
    # While $s$ has space for another element from $t$
    while $s < 2^{r_s - r_t}$:
        # Pop an element $t_\mathrm{top}$ from $t$
        $t$, $t_{\mathrm{top}}$ = stack_pop($t$)

        # and push $t_\mathrm{top}$ into the lower bits of $s$
        $s$ = $2^{r_t} \cdot s$ + $t_{\mathrm{top}}$
    return $s$, $t$
\end{lstlisting}

The condition $s < 2^{r_s - r_t}$ guarantees that $2^{r_t} \cdot s +
t_{\text{top}} < 2^{r_s}$, and thus there can be no loss of information
resulting from overflow. We also have
\begin{equation}
  \log (2^{r_t} \cdot s + t_\text{top}) \geq r_t + \log s.
\end{equation}
Applying this inequality repeatedly, once for each iteration of the while loop
in \texttt{renorm}, we have
\begin{equation}\label{eq:renorm}
\log s \geq \log s' + r_t\cdot\left[\text{\# elements popped from $t$}\right]
\end{equation}
where $s, t = \texttt{renorm}(s', t)$ as in the definition of \pop.

\subsection*{Popping in sequence}
We now directly tackle the setup described in \Cref{prob:alt}, performing a
sequence of \pop\ operations to decode a sequence of data. We suppose that we
are given some initial `message' $m_0= (s_0, t_0)$, where $2^{r_s - r_t} \leq
s_0 < 2^{r_s}$ and $t_0$ is a stack of infinite depth, modelling the `source of
bits' from the problem statement.

For $n=1\ldots N$, we let $(s_n, t_n), x_n = \pop(s_{n-1}, t_{n-1})$, where
each \pop\ operation uses the corresponding distribution $P(x_n\given
x_1,\ldots,x_{n-1})$.

Now, applying \cref{eq:inner-decoder} and \cref{eq:renorm} to the
$n^\mathrm{th}$ \pop\ gives
\begin{equation}\label{eq:recursion}
  \log s_{n+1} \geq \log s_n - h(x_n\given x_1,\ldots,x_{n-1}) + b_n + \log (1
  - \epsilon)
\end{equation}

where $b_n$ is the number of bits popped/observed from the stack $t$ in the
$n^{\mathrm{th}}$ \pop\ step. Applying \cref{eq:recursion} recursively, for
$n=1,\ldots, N$, yields
\begin{equation}
  \log s_N \geq \log s_0 - h(x_1, \ldots, x_N) + \sum_{n=1}^{N} b_n +
  N\log(1-\epsilon)
\end{equation}
which can be rearranged to give
\begin{equation}
  \sum_{n=1}^Nb_n \leq h(x_1, \ldots, x_N) - N\log(1-\epsilon) + r_t
  \approx h(x_1, \ldots, x_N) + N\epsilon + r_t
\end{equation}
since $\log(1 - \epsilon)\approx\epsilon$ for small $\epsilon$ and $\log s_N -
\log s_0 < r_t$. Thus rANS solves \cref{prob:alt}: the number of bits observed
from $t$ is `close' to $h(x_1,\ldots,x_N)$ in the sense that the difference is
no more than an additive constant, $r_t$, plus a term which grows linearly with
$N$, but with a very small coefficient $\epsilon$. In our implementation we use
$r_p = 16$ or less, and thus $\epsilon \leq \frac{2^{16 + 32 - 64}}{\ln 2}
\approx 2.2\times 10^{-5}$.

It now remains for us to describe the function $d$ and show that it satisfies
\cref{eq:inner-decoder}, as well as showing how to invert \pop\ to form an
encoder.

\subsection*{The function $d$}\label{sec:inner-decoder}
The function $d:\mathbb{N}\rightarrow\mathbb{N}\times\mathcal{A}$ must be a
bijection, and we aim for $d$ to satisfy \cref{eq:inner-decoder}, and thus
$P(x)\approx \frac{s'}{s}$.  Achieving this is actually fairly straightforward.
One way to define a bijection $d:s\mapsto(s', x)$ is to start with a mapping
$\tilde d: s\mapsto x$, with the property that none of the preimages $\tilde
d^{-1}(x):=\{n\in\mathbb{N}:\tilde d(n) = x\}$ are finite for
$x\in\mathcal{A}$. Then let $s'$ be the index of $s$ within the (ordered) set
$\tilde d^{-1}(x)$, with indices starting at $0$. Equivalently, $s'$ is the
number of integers $n$ with $0\leq n<s$ and $d(n) = x$.
% TODO: Maybe explain explicitly why this is a guaranteed bijection

With this setup, the ratio
\begin{equation}\label{eq:ratio}
  \frac{s'}{s} = \frac{\abs{\{n\in\mathbb{N}: n < s, d(n) = x\}}}{s}
\end{equation}
is the density of numbers which decode to $x$, within all the natural numbers
less $s$. For large $s$ we can ensure that this ratio is close to $P(x)$ by
setting $\tilde d$ such that numbers which decode to a symbol $x$ are
distributed \emph{within the natural numbers} with density close to $P(x)$.

To do this, we partition $\mathbb{N}$ into finite ranges of equal length, and
treat each range as a model for the interval $[0, 1]$, with sub-intervals
within $[0, 1]$ corresponding to each symbol, and the width of each
sub-interval being equal to the corresponding symbol's probability (see
\cref{fig:interval}). To be precise, the mapping $\tilde d$ can then be
expressed as a composition $\tilde d = \tilde d_2 \circ \tilde d_1$, where
$\tilde d_1$ does the partitioning described above, and $\tilde d_2$ assigns
numbers within each partition to symbols (sub-intervals). So
\begin{equation}
  \tilde d_1(s) := s \bmod 2^{r_p}.
\end{equation}
Using the shorthand $\bar{s} := \tilde d_1 (a)$, and defining
\begin{equation}
  c_j := \begin{cases}
    0                    &\quad\text{if }j=1\\
    \sum_{k=1}^{j-1} p_k &\quad\text{if }j=2,\ldots,I
  \end{cases}
\end{equation}
as the (quantized) cumulative probability of symbol $a_{j-1}$,
\begin{equation}
  \tilde d_2(\bar s) := a_i\text{ where }i := \max \{j : c_j \leq \bar s\}.
\end{equation}
\Cref{fig:interval} illustrates this mapping, with a particular probability
distribution, for the range $s = 64,\ldots, 71$.

\begin{figure}[h]
  \centering
  \drawinterval
  \caption{
    Showing the correspondence between $s$, $s \bmod 2^{r_p}$ and the symbol
    $x$. The interval $[0, 1]\subset\mathbb{R}$ is modelled by the set of
    integers $\{0, 1, \ldots, 2^{r_p} - 1\}$. In this case $r_p = 3$ and the
    probabilities of each symbol are $P(a) = \frac{1}{8}$, $P(b) =
    \frac{2}{8}$, $P(c) = \frac{3}{8}$ and $P(d) = \frac{1}{8}$.}
  \label{fig:interval}
\end{figure}

\begin{figure}[h]
  \centering
  \drawpmf \quad \drawapprox{20}{\small} \quad \drawapprox{70}{\tiny}
  \caption{
    Showing the pmf of a distribution over symbols (left) and a visualization
    of the mapping $d$ (middle and right).  Numbers less than or equal to $s$
    are shown in bold, for $s=20$ and $s=70$ (middle and right, respecitively),
    and the ratio in \cref{eq:ratio} can be seen to approach the histogram in
    the left hand plot.
  }\label{fig:visual-ans}
\end{figure}

\subsection*{Computing $s'$}
The number $s'$ was defined above as `the index of $s$ within the (ordered) set
$\tilde d^{-1}(x)$, with indices starting at $0$'. We now derive an expression
for $s'$ in terms of $s$, $p_i$ and $c_i$, where $i = \max\{j: c_j \leq \bar
s\}$ (as above), and we prove \cref{eq:inner-decoder}.

Our expression for $s'$ is a sum of two terms. The first term counts the entire
intervals, corresponding to $a_i$, which are below $s$. This is equal to $p_i
\cdot (s \div 2^{r_p})$, where $\div$ denotes \emph{integer} division,
discarding any remainder. The second term counts our position within the
current inverval, which is $\bar s - c_i \equiv s\bmod 2^{r_p} - c_i$. Thus
\begin{equation}\label{eq:s' def}
  s' = p_i \cdot (s \div 2^{r_p}) + s\bmod 2^{r_p} - c_i.
\end{equation}
This expression is straightforward to compute. Moreover from this expression it
is straightforward to prove \cref{eq:inner-decoder}. Firstly, taking the $\log$
of both sides of \cref{eq:s' def} and using the fact that $s\bmod 2^{r_p} - c_i
\geq 0$ gives
\begin{align}
  \log s' \geq \log (p_i\cdot (s\div 2^{r_p})).
\end{align}
then by the definition of $\div$, we have $s\div 2^{r_p} > \frac{s}{2^{r_p}} -
1$, and thus
\begin{align}
  \log s'
    &\geq \log\left(p_i\left(\frac{s}{2^{r_p}} -1\right)\right)\\
    &\geq \log s - h(x) + \log\left(1 - \frac{2^{r_p}}{s}\right)
\end{align}
using the fact that $P(x) = \frac{p_i}{2^{r_p}}$. Finally, using $s \geq 2^{r_s
- r_t}$,
\begin{equation}
  \log s' \geq \log s - h(x) + \log (1 - \epsilon)
\end{equation}
as required.

By choosing $r_s - r_t$ to be reasonably
large (it is equal to 32 in our implementation), we ensure that
$\frac{s'}{s}$ is very close to $P(x)$. This behaviour can be seen visually in
\cref{fig:visual-ans}, which shows the improvement in the approximation for
larger $s$.

\subsection*{Pseudocode for $d$}
We now have everything we need to write down a procedure to compute $d$. We
assume access to a function $f_P:\bar{s}\mapsto a_i, c_i, p_i$, where $i$ is
defined above. This function clearly depends on the distribution $P$, and its
computational complexity is equivalent to that of computing the CDF and inverse
CDF of $P$. For many common distributions, the CDF and inverse CDF have
straightforward closed form expressions, which don't require an explicit sum
over $i$.

We compute $d$ as follows:
\begin{lstlisting}
def $d$($s$):
    $\bar s$ = $s\bmod 2^{r_p}$
    $x$, $c$, $p$ = $f_P(\bar s)$
    $s'$ = $p \cdot (s \div 2^{r_p}) + \bar s - c$
    return $s'$, $x$
\end{lstlisting}

\subsection*{Inverting the decoder}
Having described a decoding process which appears not to throw away any
information, we now derive the inverse process, \push, and show that it is
computationally straightforward.

The \push\ function has access to the symbol $x$ as one of its inputs, and must
do two things. Firstly it must \texttt{stack\_push} the correct number of
elements to $t$ from the lower bits of $s$. Then it must reverse the effect of
$d$ on $s$, returning a value of $s$ identical to that before \pop\ was
applied.

Thus, on a high level, the inverse of the function \pop\ can be expressed as
\begin{lstlisting}
def push(($s$, $t$), $x$):
    $p$, $c$ = $g_P$($x$)
    $s'$, $t$ = renorm_inverse($s$, $t$; $p$)
    $s$ = $d^{-1}$($s'$; $p$, $c$)
    return $s$, $t$
\end{lstlisting}
where $g_P:x\mapsto p_i, c_i$ with $i$ as above, is similar to $f_P$ in that it
is analagous to computing the quantized CDF and mass function $P$. The function
$d^{-1}$ is really a pseudo-inverse of $d$; it is the inverse of $s\mapsto
d(s, x)$, holding $x$ fixed.

As mentioned above, \texttt{renorm\_inverse} must \texttt{stack\_push} the
correct amount of data from the lower order bits of $s$ into $t$. A necessary
condition which the output of \texttt{renorm} must satisfy is $2^{r_s - r_t}
\leq d^{-1}(s'; p, c) < 2^{r_s}$.

The definition of $s'$ is straightforward to invert, yielding a formula for
$d^{-1}$:
\begin{equation}
  d^{-1}(s'; p, c) = 2^{r_p} \cdot (s' \div p) + s' \bmod p + c
\end{equation}

A na\"ive implementation of \texttt{renorm\_inverse} would be
\begin{lstlisting}
def renorm_inverse($s$, $t$):
    while $d^{-1}$($s$; $p$, $c$) $\geq 2^{r_s}$:
        $t$ = stack_push($t$, $s\bmod 2^{r_t}$)
        $s$ = $s\div 2^{r_t}$
    return $s$, $t$
\end{lstlisting}
While $d^{-1}(s', x) < 2^{r_s}$ is certainly a \emph{necessary} condition for
the correctness of \texttt{renorm\_inverse}, it isn't immediately clear whether
it is sufficient. In particular, it is possible that the first $s'$ for which
$d^{-1}(s', p, c) < 2^{r_s}$ is not the correct one. We can verify that $s'$ is
correct by ensuring that $d^{-1}(s'\div2^{r_t}) < 2^{r_s - r_t}$, which would
imply that $s'$ is the only possible value.
% TODO: make this much clearer


\printbibliography
\end{document}
