\documentclass{article}

\usepackage{hyperref}

\usepackage{color}

\usepackage{tikz}
\usetikzlibrary{calc}
\usetikzlibrary{math}

\newcommand{\barwidth}{.5cm}

\newcommand{\drawaxes}[1]{
  \draw [->] (-.2, 0) -> (4 * \barwidth + .2cm, 0) node[anchor=west] {$x$};
  \draw [->] (-.2, 0) -> (-.2, 5.2) node[anchor=south] {$#1$};
  \foreach \x in {0, ..., 4}
    \draw (\x * \barwidth, -1pt) -- (\x * \barwidth, 1pt);
  \foreach \x/\char in {0.5/$a$, 1.5/$b$, 2.5/$c$, 3.5/$d$}
    \draw (\x * \barwidth, 0) node[anchor=north, text height=8] {\char};
}

\newcommand{\drawbar}[3]{
  \draw[#3] (#1, 0) -- (#1, 12.5cm * #2) -- (\barwidth + #1, 12.5cm * #2) --
  (\barwidth + #1, 0);
}

\newcommand{\drawpmf}{
\begin{tikzpicture}
  \drawaxes{P(x)}
  \foreach \y in {0.125, 0.25, 0.375}
    \draw (-.2cm - 1pt, 12.5 * \y) node[anchor=east] {$\y$}
      --  (-.2cm + 1pt, 12.5 * \y);
  \drawbar{0 * \barwidth}{0.125}{};
  \drawbar{1 * \barwidth}{0.25 }{};
  \drawbar{2 * \barwidth}{0.375}{};
  \drawbar{3 * \barwidth}{0.25 }{};
\end{tikzpicture}
}

\newcommand{\drawapprox}[2]{
\begin{tikzpicture}
  \tikzmath{
    \splus = #1 + 1;
    \yunit = 12.5 / \splus;
  }
  \draw[xstep=\barwidth, ystep=\yunit, lightgray, very thin]
    (0, 0) grid (4 * \barwidth, 5.2);
    \drawaxes{s'}
  \foreach \y [parse=true, count=\s] in {\yunit, 2 * \yunit, ..., 5.2}{
    \node (tick) at (-.2cm, \y) {};
    \draw ($(tick) - (1pt, 0)$) -- ($(tick) + (1pt, 0)$);
    \draw ($(tick) - (0, \yunit / 2) - (.05, 0)$) node[anchor=east]
      {\pgfmathparse{int(\s - 1)} #2 $\pgfmathresult$};
  }
  \foreach \x/\freq/\start in {0/1/0, 1/2/1, 2/3/3, 3/2/6}{
    \foreach \y [parse=true, count=\s] in {\yunit, 2 * \yunit, ..., 5.2}{
      \tikzmath{
        \snew = int(8 * int((\s - 1) / \freq) + mod((\s - 1), \freq) + \start);
        if \snew>#1 then {\toprint = \snew;} else {let \toprint = ;};
        }
      \draw ($6.25/\splus*(0, -1) + 0.45*(\barwidth,0)
      + \s*(0, 12.5 / \splus) + \x*(\barwidth, 0)$) node [text=lightgray] {#2 $\toprint$};
    }
  }
  \drawbar{0 * \barwidth}{0.125}{thin, gray};
  \drawbar{1 * \barwidth}{0.25} {thin, gray};
  \drawbar{2 * \barwidth}{0.375}{thin, gray};
  \drawbar{3 * \barwidth}{0.25} {thin, gray};
  \foreach \x/\freq/\start in {0/1/0, 1/2/1, 2/3/3, 3/2/6}{
    \foreach \y [parse=true, count=\s] in {\yunit, 2 * \yunit, ..., 5.2}{
      \tikzmath{
        \snew = int(8 * int((\s - 1) / \freq) + mod((\s - 1), \freq) + \start);
        if \snew<=#1 then {\toprint = \snew;} else {let \toprint = ;};
        }
      \draw ($(0, -\yunit / 2) + 0.47*(\barwidth,0)
      + \s*(0, \yunit) + \x*(\barwidth, 0)$) node {#2 $\mathbf\toprint$};
    }
  }
\end{tikzpicture}
}


\usepackage{booktabs}

\usepackage{amsmath}
\usepackage{amssymb}
\usepackage{mathtools}
\newcommand\given{\,\vert\,}
\DeclarePairedDelimiter\abs{\lvert}{\rvert}

\usepackage[style=authoryear, date=year, natbib=true, maxcitenames=2,
            maxbibnames=10, uniquelist=false]{biblatex}
\setlength\bibitemsep{1.5\itemsep}
\DeclareFieldFormat[article, book, inbook, incollection, inproceedings, misc,
                    thesis, unpublished, techreport, phdthesis]{title}{#1}
\DeclareNameAlias{sortname}{family-given}
\addbibresource{ans-notes.bib}
\renewcommand*{\intitlepunct}{\space}

\usepackage{siunitx}
\DeclareSIUnit[parse-numbers=false]\bit{bits}
\newcommand\bits{\,\si\bit}

\usepackage{listings}
\lstset{language=Python, mathescape=true, columns=flexible, xleftmargin=0.9cm}

\usepackage{cleveref}
\newtheorem{problem}{Problem}

\author{James Townsend}
\title{Notes on asymmetric numeral systems}

\begin{document}
\maketitle
\section{Introduction}
We are interested in algorithms for lossless compression of sequences of data.
The range variant of asymmetric numeral systems (ANS) is such an algorithm,
and, like arithmetic coding (AC), it is close to optimal in terms of
compression rate \citep{dudaAsymmetricNumeralSystems2009}. The key difference
between ANS and AC is that ANS is last-in-first-out (LIFO), while AC is
first-in-first-out (FIFO).

ANS comprises two functions, which we denote $C$ and $D$, for encoding and
decoding, respectively. $C$ accepts some pre-compressed information $m$, and a
symbol $x$ to be compressed, and returns a piece of compressed information,
$m'$. Thus it has the signiature

\begin{equation}
  C:(m, x) \mapsto m'.
\end{equation}

The new compressed state, $m'$, contains precisely the same information as the
pair $(m, x)$, and therefore $C$ can be inverted, giving a decoder mapping. The
decoder, $D$, maps from $m'$ back to $m, x$:

\begin{equation}
  D:m' \mapsto (m, x).
\end{equation}

$C$ and $D$ are inverse to one another, so $C(D(m)) = m$ and $D(C(m,
x)) = (m, x)$. Encoding and decoding require computation of a mass function $P$
over symbols, and when we want to be explicit about this we use $C_P$ and $D_P$
for encoding and decoding (resp.) according to $P$.

%TODO: intro more/better notation for prob distro

\section{Building an encoder/decoder pair}
The problem which the ANS encoder $C$ solves is

\begin{problem}\label{prob:default}
        Given a sequence of random variables $X_1, \ldots, X_T$, find an
        invertible algorithm $C$ which will map any sample $x_1, \ldots, x_T$
        to a binary message $b$, such that the length of $b$ is close to the
        information content $h(x_1, \ldots, x_T)$.
\end{problem}

Since $C$ is invertible, we can reformulate \Cref{prob:default} in terms of
$C$'s inverse, $D$. This leads to a different, but equivalent, problem
statement:

\begin{problem}\label{prob:alt}
        Given a sequence of random variables $X_1, \ldots, X_T$, find an
        invertible algorithm $D$ which maps a source of bits to a sequence
        $x_1, \ldots, x_T$, such that the number of bits observed is close to
        the information content $h(x_1, \ldots, x_T)$.
\end{problem}

Our description of the details of ANS focuses on the decoder, because this
leads to a more straightforward presentation. Having described $D$ and shown
that it solves \Cref{prob:alt}, we then show how to invert it to form $C$.

% TODO: We also describe the inverse procedure, checking that it is
% computationally tractible.

\subsection*{The structure of $m$}
We use a pair $m = (s, t)$ as the data structure for $m$. The element $t$ is a
stack of unsigned integers of some fixed precision $r_t$. This stack has push
and pop operations. The element $s$ is an unsigned integer with precision $r_s$
where $r_s > r_t$. In our implementation we use $r_t = 32$ and $r_s = 64$. The
stack $t$, along with its pop operation, can model the `source of bits' from
our problem statement above.
% TODO: Diagram of $m$

\subsection*{How $D$ works}
Our strategy for performing a decode $D$ will be to firstly extract a symbol
from $s$. We do this using a bijective function $d:\mathbb N\rightarrow\mathbb
N\times\mathcal{A}$, which takes an integer $s$ as input and returns a pair
$(s', x)$, where $s'$ is an integer and $x$ is a symbol. We design $d$ so that

\begin{equation}\label{eq:inner-decoder}
\log s' \approx \log s - h(x).
\end{equation}

See below for details on $d$, where we make precise the
approximation in \cref{eq:inner-decoder} and show that for large $s$ the
approximation is extremely accurate.

We require that the approximation \cref{eq:inner-decoder} is accurate, and thus
that $s$ remains sufficiently `large'. Thus, after extracting a symbol using
$d$, we check whether $s$ is below a certain threshold, and if it is we pop an
integer from $t$ and move its contents into the lower order bits of $s$,
increasing the size of $s$. We refer to this as `renormalization'.  To be
precise, we apply the following function:

\begin{lstlisting}
def renorm($s$, $t$):
    while $s < 2^{r_s - r_t}$:
        $s_{\text{new}}$, $t$ = pop($t$)
        $s$ = $2^{r_t} \cdot s$ + $s_{\text{new}}$
    return $s$, $t$
\end{lstlisting}

The condition $s < 2^{r_s - r_t}$ guarantees that $2^{r_t} \cdot s +
s_{\text{new}} < 2^{r_s}$, and thus there is no loss of information resulting
from overflow. In the case $s \geq 2^{r_s - r_t}$, clearly $\text{renorm}(s, t)
\equiv (s, t)$. For the case $s < 2^{r_s - r_t}$, we have
\begin{align}
        \log (2^{r_t} \cdot s + s_\text{new})
        &= r_t + \log s + \log\left(1 + \frac{s_\text{new}}{2^{r_t} \cdot
                s}\right)\\
        &\leq r_t + \log s + \frac{s_\text{new}}{2^{r_t} \cdot s\ln 2}\\
        &\leq r_t + \log s + \epsilon
\end{align}
where $\epsilon := \frac{1}{s\ln 2}$. This implies that the quantity
\begin{equation}
\log s - r_t\cdot\left[\text{\# elements popped from $t$}\right]
\end{equation}
is approximately preserved by renorm.
%TODO: more detail here.

Using $d$ and our $renorm$ function, we can describe the whole of the decoder
$D$:

\begin{lstlisting}
def $D$($s$, $t$):
    $s$, $x$   $=$ $d$($s$)
    $s$, $t$   $=$ renorm($s$, $t$)
    return ($s$, $t$), $x$
\end{lstlisting}
If $(s', t'), x = D(s, t)$, then we have
\begin{equation}\label{eq:ans-approx}
        \log s' + h(x) - r_t\cdot\left[\text{\# elements popped from $t$}\right]
        \approx \log s.
\end{equation}
If we apply $D$ many times to decode a sequence of symbols,
\cref{eq:ans-approx} can be applied recursively. Denote the decoded symbols
$x_1, \ldots, x_N$, $s, t$ the coder state before decoding and $s', t'$ the
state after decoding, then
\begin{equation}
        \log s' + \sum_{n=1}^Nh(x_n) - r_t\cdot\left[\text{\# elements popped
        from $t$}\right] \approx \log s.
\end{equation}
and in particular
\begin{align}
         r_t\cdot\left[\text{\# elements popped
         from $t$}\right] &\approx \sum_{n=1}^Nh(x_n) + \log s' - \log s\\
                          &\leq \sum_{n=1}^Nh(x_n) + r_t
\end{align}
%TODO: convert from \approx to = with epsilon
which implies that the scheme solves the original problem statement.

\subsection*{The function $d$}\label{sec:inner-decoder}
The function $d:\mathbb{N}\rightarrow\mathbb{N}\times\mathcal{A}$ must be a
bijection. Bijections of this form can be visualised using Cartesian axes, with
the elements of $\mathcal{A}$ on the horizontal axis and the elements of
$\mathbb{N}$ on the vertical. Each element $s\in\mathbb{N}$ is then positioned
in two dimensional space according to the coordinates given by $s', x = d(s)$.
The middle and right plots in \cref{fig:visual-ans} show the mapping $d$ for a
particular probability distribution $P$ which is shown in the left plot.
\begin{figure}[t]
  \centering
  \drawpmf \quad \drawapprox{20}{\small} \quad \drawapprox{70}{\tiny}
  \caption{
    Showing the pmf of a distribution over symbols (left) and the
    approximations to that distribution implied by ANS for $s=20$ and $s=70$
    (middle and right, respectively).}\label{fig:visual-ans}
\end{figure}

We aim for $d$ to satisfy \cref{eq:ans-approx}, and thus $P(x)\approx
\frac{s'}{s}$. This can be achieved by ensuring that numbers which decode to $x$
are distributed \emph{within the natural numbers} with density close to $P(x)$.

\printbibliography
\end{document}
